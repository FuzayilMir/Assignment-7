\documentclass[a4paper,12pt]{article}
\usepackage{graphicx}
\usepackage{amsmath}
\usepackage{tkz-euclide}
\usepackage{booktabs}
\title{Assignment-7\\ Latex Report}
\author{Fuzayil Bin Afzal Mir}
\date{24/01/2021}
\begin{document}
	\maketitle
	

 \newpage
 \begin{itemize}
	    \item \Large\textbf{Exercise 2.29}
	\end{itemize}
	\section{With the same centre O, draw two circles of radii 4 and 2.5.}\\
    	
\subsection{Solution} \\
It is given that,
O is the centre\\ 
 Let,'A' be a point on the boundary of a circle having radius 2.5 cm and 
  Let, 'B' be a point on the boundary of the circle having radius 4cm respectively.
 \\Therefore,\\ OA =2.5 cm\\
 and\\ OB = 4 cm\\
  
  
  


\begin{tikzpicture}[scale=1,point/.style={draw,circle,fill = gray,inner sep = 1pt}]
   \begin{center}
  
       \draw(0,0) circle(2.5cm);
        \node (O) at (0,0)[point,label=right :$O$] {};
         \node (A) at (-2.5cm,0)[point,label=left :$A$] {};
       
         
         \draw (A) -- node[above] {$\textrm{2.5}$}
         
       \draw(0,0) circle(4cm);
         \node (B) at (0,-4cm)[point,label=above left :$B$] {};
         \draw (B) -- node[right] {$\textrm{4}$}
   \end{center}
\end{tikzpicture}
\\

\textbf{\underline{Note:}} {Figure generated using latex}\\












\subsection{Figure of the circles with common centre O}
\begin{figure}
    \centering
    \includegraphics[width=10cm]{fig.png}
    \caption{Generated using python}
    \label{fig:2}
\end{figure}

\\$$$$\\

  \textbf{Download the python code used for generating the figure from here:}
 
 \fbox{
  \begin{lstlisting}
https://github.com/FuzayilMir/Assignment-7/blob/main/assignment-7fig.py
 \end{lstlisting}
 
 
 

}



















\end{document}
